\documentclass[12pt,francais]{report}

\usepackage[UTF8]{inputenc}
\usepackage[pdftex]{graphicx}
\usepackage[french]{babel}
\usepackage{enumitem}

\begin{document}

\begin{center}
\includegraphics[scale=0.3]{./images/logo_universite_orleans.png}~\\[1.5cm]

\textsc{\large Master 1 - TER}\\[1cm]	

\rule{\linewidth}{0.5mm}\\[0.4cm]

{ \LARGE \bfseries GoodLua\\[0.4cm] }

\rule{\linewidth}{0.5mm}\\[2cm]
\end{center}


\begin{minipage}{0.4\textwidth}
      \begin{flushleft}
			\textsc{Bovie} Pierre-Edouard\\
			\textsc{Labourbe} Loïc\\
			\textsc{Maslowski} Antoine\\
			\textsc{Roche} Julie\\
			\end{flushleft}
    \end{minipage}\hfill
    \begin{minipage}{0.6\textwidth}
      \begin{flushright}
			Enseignants : \textsc{Couvreur} Jean-Michel\\
			\textsc{Dabrowski} Frederic\\

\end{flushright}
    \end{minipage}\\[1.5cm]
		
		\begin{center}
    {7 Février 2018 - 25 Mai 2018}
		\end{center}

\chapter*{Besoins fonctionnels}

\begin{itemize}[label=\textbullet]
	\item \textbf{exécuter du code}
	\begin{description}
		\item \textit{description :} exécuter du code eLua sur le PcDuino
		\item \textit{priorité :} très haute
		\item \textit{justification :} cette fonctionnalité est la principale. Nous devons être en mesure de faire fonctionner le robot grâce à du code eLua avant toute autre chose.
	\end{description}
	\item \textbf{connecter android au robot grâce au wifi}
	\begin{description}
		\item \textit{description :} utiliser la carte wifi du PcDuino pour la connecter à un appareil androïd.
		\item \textit{priorité :} haute
		\item \textit{justification :} le code crée via l'application androïd doit être transmis au robot pour être exécuté. Une solution filaire étant peu pratique, nous le connecterons via wifi.
	\end{description}
	\item \textbf{transformer blockly en eLua}
	\begin{description}
		\item \textit{description :} traduire les blocs de la librairie Blockly dans le langage Lua, plus spécifiquement en eLua.
		\item \textit{priorité :} moyenne
		\item \textit{justification :} le robot étant "guidé" par du code eLua, il est nécessaire de procéder à une traduction du programme visuel dans ce langage avant de l'envoyer au robot pour être exécuté.
	\end{description}
	\item \textbf{créer un programme}
	\begin{description}
		\item \textit{description :} la création d'un nouveau programme blockly via l'application androïd.
		\item \textit{priorité :} moyenne
		\item \textit{justification :} c'est la fonctionnalité de base de l'application, il faut pouvoir créer de nouveaux programmes en vue de les écrire et de les exécuter.
	\end{description}
	\item \textbf{sauvegarder / charger des programmes}
	\begin{description}
		\item \textit{description :} l'utilisateur pourra sauvegarder les programmes sur lesquels il travaille, et reprendre sa progression là où il était en chargeant un programme parmi ceux enregistrés.
		\item \textit{priorité :} basse
		\item \textit{justification :} cela permettra de travailler sur un programme en plusieurs fois sans risque de perdre son travail. Cette fonctionnalité était à la base en option, mais l'architecture de notre application nécessite de pouvoir l'utiliser.
	\end{description}
\end{itemize}


\chapter*{Besoins non fonctionnels}

\begin{description}
	\item [environnement de travail :]
	\item [exécution en temps réel / rapidité :]
	\item [simplicité d'utilisation : ]
\end{description}


\chapter*{Analyse de l'existant}

Blockly : \cite{ref2}

Composants du robot : \cite{ref1}


\bibliographystyle{unsrt}
\bibliography{biblio}

\end{document}
